% Generated by Sphinx.
\def\sphinxdocclass{report}
\documentclass[letterpaper,10pt,english]{sphinxmanual}
\usepackage[utf8]{inputenc}
\DeclareUnicodeCharacter{00A0}{\nobreakspace}
\usepackage{cmap}
\usepackage[T1]{fontenc}
\usepackage{babel}
\usepackage{times}
\usepackage[Bjarne]{fncychap}
\usepackage{longtable}
\usepackage{sphinx}
\usepackage{multirow}

\addto\captionsenglish{\renewcommand{\figurename}{Fig. }}
\addto\captionsenglish{\renewcommand{\tablename}{Table }}
\floatname{literal-block}{Listing }



\title{PyGEOS Documentation}
\date{September 07, 2016}
\release{1.0}
\author{Eric Daniels}
\newcommand{\sphinxlogo}{}
\renewcommand{\releasename}{Release}
\makeindex

\makeatletter
\def\PYG@reset{\let\PYG@it=\relax \let\PYG@bf=\relax%
    \let\PYG@ul=\relax \let\PYG@tc=\relax%
    \let\PYG@bc=\relax \let\PYG@ff=\relax}
\def\PYG@tok#1{\csname PYG@tok@#1\endcsname}
\def\PYG@toks#1+{\ifx\relax#1\empty\else%
    \PYG@tok{#1}\expandafter\PYG@toks\fi}
\def\PYG@do#1{\PYG@bc{\PYG@tc{\PYG@ul{%
    \PYG@it{\PYG@bf{\PYG@ff{#1}}}}}}}
\def\PYG#1#2{\PYG@reset\PYG@toks#1+\relax+\PYG@do{#2}}

\expandafter\def\csname PYG@tok@si\endcsname{\let\PYG@it=\textit\def\PYG@tc##1{\textcolor[rgb]{0.44,0.63,0.82}{##1}}}
\expandafter\def\csname PYG@tok@k\endcsname{\let\PYG@bf=\textbf\def\PYG@tc##1{\textcolor[rgb]{0.00,0.44,0.13}{##1}}}
\expandafter\def\csname PYG@tok@nl\endcsname{\let\PYG@bf=\textbf\def\PYG@tc##1{\textcolor[rgb]{0.00,0.13,0.44}{##1}}}
\expandafter\def\csname PYG@tok@sc\endcsname{\def\PYG@tc##1{\textcolor[rgb]{0.25,0.44,0.63}{##1}}}
\expandafter\def\csname PYG@tok@s\endcsname{\def\PYG@tc##1{\textcolor[rgb]{0.25,0.44,0.63}{##1}}}
\expandafter\def\csname PYG@tok@kc\endcsname{\let\PYG@bf=\textbf\def\PYG@tc##1{\textcolor[rgb]{0.00,0.44,0.13}{##1}}}
\expandafter\def\csname PYG@tok@nf\endcsname{\def\PYG@tc##1{\textcolor[rgb]{0.02,0.16,0.49}{##1}}}
\expandafter\def\csname PYG@tok@c\endcsname{\let\PYG@it=\textit\def\PYG@tc##1{\textcolor[rgb]{0.25,0.50,0.56}{##1}}}
\expandafter\def\csname PYG@tok@vi\endcsname{\def\PYG@tc##1{\textcolor[rgb]{0.73,0.38,0.84}{##1}}}
\expandafter\def\csname PYG@tok@gr\endcsname{\def\PYG@tc##1{\textcolor[rgb]{1.00,0.00,0.00}{##1}}}
\expandafter\def\csname PYG@tok@gi\endcsname{\def\PYG@tc##1{\textcolor[rgb]{0.00,0.63,0.00}{##1}}}
\expandafter\def\csname PYG@tok@cp\endcsname{\def\PYG@tc##1{\textcolor[rgb]{0.00,0.44,0.13}{##1}}}
\expandafter\def\csname PYG@tok@nd\endcsname{\let\PYG@bf=\textbf\def\PYG@tc##1{\textcolor[rgb]{0.33,0.33,0.33}{##1}}}
\expandafter\def\csname PYG@tok@gt\endcsname{\def\PYG@tc##1{\textcolor[rgb]{0.00,0.27,0.87}{##1}}}
\expandafter\def\csname PYG@tok@s2\endcsname{\def\PYG@tc##1{\textcolor[rgb]{0.25,0.44,0.63}{##1}}}
\expandafter\def\csname PYG@tok@vc\endcsname{\def\PYG@tc##1{\textcolor[rgb]{0.73,0.38,0.84}{##1}}}
\expandafter\def\csname PYG@tok@nc\endcsname{\let\PYG@bf=\textbf\def\PYG@tc##1{\textcolor[rgb]{0.05,0.52,0.71}{##1}}}
\expandafter\def\csname PYG@tok@w\endcsname{\def\PYG@tc##1{\textcolor[rgb]{0.73,0.73,0.73}{##1}}}
\expandafter\def\csname PYG@tok@sb\endcsname{\def\PYG@tc##1{\textcolor[rgb]{0.25,0.44,0.63}{##1}}}
\expandafter\def\csname PYG@tok@ni\endcsname{\let\PYG@bf=\textbf\def\PYG@tc##1{\textcolor[rgb]{0.84,0.33,0.22}{##1}}}
\expandafter\def\csname PYG@tok@na\endcsname{\def\PYG@tc##1{\textcolor[rgb]{0.25,0.44,0.63}{##1}}}
\expandafter\def\csname PYG@tok@kn\endcsname{\let\PYG@bf=\textbf\def\PYG@tc##1{\textcolor[rgb]{0.00,0.44,0.13}{##1}}}
\expandafter\def\csname PYG@tok@bp\endcsname{\def\PYG@tc##1{\textcolor[rgb]{0.00,0.44,0.13}{##1}}}
\expandafter\def\csname PYG@tok@sh\endcsname{\def\PYG@tc##1{\textcolor[rgb]{0.25,0.44,0.63}{##1}}}
\expandafter\def\csname PYG@tok@mf\endcsname{\def\PYG@tc##1{\textcolor[rgb]{0.13,0.50,0.31}{##1}}}
\expandafter\def\csname PYG@tok@mb\endcsname{\def\PYG@tc##1{\textcolor[rgb]{0.13,0.50,0.31}{##1}}}
\expandafter\def\csname PYG@tok@m\endcsname{\def\PYG@tc##1{\textcolor[rgb]{0.13,0.50,0.31}{##1}}}
\expandafter\def\csname PYG@tok@vg\endcsname{\def\PYG@tc##1{\textcolor[rgb]{0.73,0.38,0.84}{##1}}}
\expandafter\def\csname PYG@tok@kp\endcsname{\def\PYG@tc##1{\textcolor[rgb]{0.00,0.44,0.13}{##1}}}
\expandafter\def\csname PYG@tok@c1\endcsname{\let\PYG@it=\textit\def\PYG@tc##1{\textcolor[rgb]{0.25,0.50,0.56}{##1}}}
\expandafter\def\csname PYG@tok@o\endcsname{\def\PYG@tc##1{\textcolor[rgb]{0.40,0.40,0.40}{##1}}}
\expandafter\def\csname PYG@tok@go\endcsname{\def\PYG@tc##1{\textcolor[rgb]{0.20,0.20,0.20}{##1}}}
\expandafter\def\csname PYG@tok@gu\endcsname{\let\PYG@bf=\textbf\def\PYG@tc##1{\textcolor[rgb]{0.50,0.00,0.50}{##1}}}
\expandafter\def\csname PYG@tok@il\endcsname{\def\PYG@tc##1{\textcolor[rgb]{0.13,0.50,0.31}{##1}}}
\expandafter\def\csname PYG@tok@kd\endcsname{\let\PYG@bf=\textbf\def\PYG@tc##1{\textcolor[rgb]{0.00,0.44,0.13}{##1}}}
\expandafter\def\csname PYG@tok@cs\endcsname{\def\PYG@tc##1{\textcolor[rgb]{0.25,0.50,0.56}{##1}}\def\PYG@bc##1{\setlength{\fboxsep}{0pt}\colorbox[rgb]{1.00,0.94,0.94}{\strut ##1}}}
\expandafter\def\csname PYG@tok@sr\endcsname{\def\PYG@tc##1{\textcolor[rgb]{0.14,0.33,0.53}{##1}}}
\expandafter\def\csname PYG@tok@se\endcsname{\let\PYG@bf=\textbf\def\PYG@tc##1{\textcolor[rgb]{0.25,0.44,0.63}{##1}}}
\expandafter\def\csname PYG@tok@gs\endcsname{\let\PYG@bf=\textbf}
\expandafter\def\csname PYG@tok@cm\endcsname{\let\PYG@it=\textit\def\PYG@tc##1{\textcolor[rgb]{0.25,0.50,0.56}{##1}}}
\expandafter\def\csname PYG@tok@ss\endcsname{\def\PYG@tc##1{\textcolor[rgb]{0.32,0.47,0.09}{##1}}}
\expandafter\def\csname PYG@tok@err\endcsname{\def\PYG@bc##1{\setlength{\fboxsep}{0pt}\fcolorbox[rgb]{1.00,0.00,0.00}{1,1,1}{\strut ##1}}}
\expandafter\def\csname PYG@tok@nb\endcsname{\def\PYG@tc##1{\textcolor[rgb]{0.00,0.44,0.13}{##1}}}
\expandafter\def\csname PYG@tok@ne\endcsname{\def\PYG@tc##1{\textcolor[rgb]{0.00,0.44,0.13}{##1}}}
\expandafter\def\csname PYG@tok@ow\endcsname{\let\PYG@bf=\textbf\def\PYG@tc##1{\textcolor[rgb]{0.00,0.44,0.13}{##1}}}
\expandafter\def\csname PYG@tok@mh\endcsname{\def\PYG@tc##1{\textcolor[rgb]{0.13,0.50,0.31}{##1}}}
\expandafter\def\csname PYG@tok@gd\endcsname{\def\PYG@tc##1{\textcolor[rgb]{0.63,0.00,0.00}{##1}}}
\expandafter\def\csname PYG@tok@gp\endcsname{\let\PYG@bf=\textbf\def\PYG@tc##1{\textcolor[rgb]{0.78,0.36,0.04}{##1}}}
\expandafter\def\csname PYG@tok@ge\endcsname{\let\PYG@it=\textit}
\expandafter\def\csname PYG@tok@sx\endcsname{\def\PYG@tc##1{\textcolor[rgb]{0.78,0.36,0.04}{##1}}}
\expandafter\def\csname PYG@tok@kt\endcsname{\def\PYG@tc##1{\textcolor[rgb]{0.56,0.13,0.00}{##1}}}
\expandafter\def\csname PYG@tok@gh\endcsname{\let\PYG@bf=\textbf\def\PYG@tc##1{\textcolor[rgb]{0.00,0.00,0.50}{##1}}}
\expandafter\def\csname PYG@tok@no\endcsname{\def\PYG@tc##1{\textcolor[rgb]{0.38,0.68,0.84}{##1}}}
\expandafter\def\csname PYG@tok@kr\endcsname{\let\PYG@bf=\textbf\def\PYG@tc##1{\textcolor[rgb]{0.00,0.44,0.13}{##1}}}
\expandafter\def\csname PYG@tok@nv\endcsname{\def\PYG@tc##1{\textcolor[rgb]{0.73,0.38,0.84}{##1}}}
\expandafter\def\csname PYG@tok@s1\endcsname{\def\PYG@tc##1{\textcolor[rgb]{0.25,0.44,0.63}{##1}}}
\expandafter\def\csname PYG@tok@nt\endcsname{\let\PYG@bf=\textbf\def\PYG@tc##1{\textcolor[rgb]{0.02,0.16,0.45}{##1}}}
\expandafter\def\csname PYG@tok@nn\endcsname{\let\PYG@bf=\textbf\def\PYG@tc##1{\textcolor[rgb]{0.05,0.52,0.71}{##1}}}
\expandafter\def\csname PYG@tok@mi\endcsname{\def\PYG@tc##1{\textcolor[rgb]{0.13,0.50,0.31}{##1}}}
\expandafter\def\csname PYG@tok@sd\endcsname{\let\PYG@it=\textit\def\PYG@tc##1{\textcolor[rgb]{0.25,0.44,0.63}{##1}}}
\expandafter\def\csname PYG@tok@mo\endcsname{\def\PYG@tc##1{\textcolor[rgb]{0.13,0.50,0.31}{##1}}}

\def\PYGZbs{\char`\\}
\def\PYGZus{\char`\_}
\def\PYGZob{\char`\{}
\def\PYGZcb{\char`\}}
\def\PYGZca{\char`\^}
\def\PYGZam{\char`\&}
\def\PYGZlt{\char`\<}
\def\PYGZgt{\char`\>}
\def\PYGZsh{\char`\#}
\def\PYGZpc{\char`\%}
\def\PYGZdl{\char`\$}
\def\PYGZhy{\char`\-}
\def\PYGZsq{\char`\'}
\def\PYGZdq{\char`\"}
\def\PYGZti{\char`\~}
% for compatibility with earlier versions
\def\PYGZat{@}
\def\PYGZlb{[}
\def\PYGZrb{]}
\makeatother

\renewcommand\PYGZsq{\textquotesingle}

\begin{document}

\maketitle
\tableofcontents
\phantomsection\label{index::doc}


PyGEOS is a Python Package designed to support common processes in the geosciences.
These tasks include interfacing with open source Python packages, GSLIB, Paraview, and Geosoft.
The goal is to simplify common tasks that require batch processing or functionality not available in commercial software.
\begin{description}
\item[{Dependencies:}] \leavevmode
PyGEOS has been developed to run on Python 3.4 and includes a number of functions from pre-existing packages.
To avoid issue be sure to install Pandas, NumPy, PyGSLIB,  and Matplotlib

\end{description}

Contents:


\chapter{PA module}
\label{index:pygeos-documentation}\label{index:pa-module}\label{index:module-pygeos.PA}\index{pygeos.PA (module)}\index{zscore() (in module pygeos.PA)}

\begin{fulllineitems}
\phantomsection\label{index:pygeos.PA.zscore}\pysiglinewithargsret{\code{pygeos.PA.}\bfcode{zscore}}{\emph{inputDF}}{}
Return a DataFrame of Zscore values
\begin{quote}\begin{description}
\item[{Parameters}] \leavevmode
\textbf{\texttt{inputDF}} -- Pandas Dataframe of values to be returned as zscore

\end{description}\end{quote}

Returns:

\end{fulllineitems}



\chapter{io module}
\label{index:module-pygeos.io}\label{index:io-module}\index{pygeos.io (module)}
module for import/export functions

NOTE: many functions require the Geosoft GX plug-in
Currently commented out b/c I am no longer working with geosoft regularly.
\index{read\_GSLIB() (in module pygeos.io)}

\begin{fulllineitems}
\phantomsection\label{index:pygeos.io.read_GSLIB}\pysiglinewithargsret{\code{pygeos.io.}\bfcode{read\_GSLIB}}{\emph{FilePath}, \emph{griddef}}{}
Read GSLIB gridded file with grid definition in header.
\begin{quote}\begin{description}
\item[{Parameters}] \leavevmode\begin{itemize}
\item {} 
\textbf{\texttt{FilePath}} -- Location of .dat GSLIB file

\item {} 
\textbf{\texttt{griddef}} -- pyGSLIB gs.GridDef object

\end{itemize}

\item[{Returns}] \leavevmode

tuple containing:

\begin{DUlineblock}{0em}
\item[] griddef: pyGSLIB griddef object with grid definition read from header
\item[] data: Pandas DataFrame with Grid(s)
\end{DUlineblock}


\item[{Return type}] \leavevmode
(tuple)

\end{description}\end{quote}

\end{fulllineitems}

\index{read\_LAS() (in module pygeos.io)}

\begin{fulllineitems}
\phantomsection\label{index:pygeos.io.read_LAS}\pysiglinewithargsret{\code{pygeos.io.}\bfcode{read\_LAS}}{\emph{FilePath}}{}
Read LAS files to pandas Dataframe and dictionary of tops
\begin{quote}\begin{description}
\item[{Parameters}] \leavevmode
\textbf{\texttt{Filepath}} -- location of LAS file

\item[{Returns}] \leavevmode

tuple containing:

\begin{DUlineblock}{0em}
\item[] TopsDict: dictionary containing tops
\item[] WellData: Pandas DataFrame with log data
\item[] 
\end{DUlineblock}

NOTE: Not all LAS files are the same, this function was designed using Petra exported files.
LAS files from other software may yield and error.


\item[{Return type}] \leavevmode
(tuple)

\end{description}\end{quote}

\end{fulllineitems}

\index{read\_g2d() (in module pygeos.io)}

\begin{fulllineitems}
\phantomsection\label{index:pygeos.io.read_g2d}\pysiglinewithargsret{\code{pygeos.io.}\bfcode{read\_g2d}}{\emph{FilePath}}{}
Read Pangeos g2d file to PyGSLIB GridDef object
\begin{quote}\begin{description}
\item[{Parameters}] \leavevmode
\textbf{\texttt{Filepath}} -- location of .g2d file

\item[{Returns}] \leavevmode
PyGSLIB griddef object

\item[{Return type}] \leavevmode
(gs.GridDef)

\end{description}\end{quote}

\end{fulllineitems}

\index{read\_gxgrid() (in module pygeos.io)}

\begin{fulllineitems}
\phantomsection\label{index:pygeos.io.read_gxgrid}\pysiglinewithargsret{\code{pygeos.io.}\bfcode{read\_gxgrid}}{\emph{GridPath}, \emph{name}}{}
Read geosoft grid to pandas dataframe
\begin{quote}\begin{description}
\item[{Parameters}] \leavevmode\begin{itemize}
\item {} 
\textbf{\texttt{GridPath}} -- Filepath of Geosoft grid

\item {} 
\textbf{\texttt{name}} -- Name for column header in pandas dataframe

\end{itemize}

\item[{Returns}] \leavevmode
Pandas DataFrame

\end{description}\end{quote}

\end{fulllineitems}

\index{read\_ply() (in module pygeos.io)}

\begin{fulllineitems}
\phantomsection\label{index:pygeos.io.read_ply}\pysiglinewithargsret{\code{pygeos.io.}\bfcode{read\_ply}}{\emph{filepath}}{}
Read polygon from .ply file
\begin{quote}\begin{description}
\item[{Parameters}] \leavevmode
\textbf{\texttt{filepath}} -- filepath for polygon file

\item[{Returns}] \leavevmode

tuple containing:

\begin{DUlineblock}{0em}
\item[] ply: pandas DataFrame containing XY coordinates for verticies
\item[] patch:  matplotlib patch for use in plotting
\end{DUlineblock}


\item[{Return type}] \leavevmode
(tuple)

\end{description}\end{quote}

\end{fulllineitems}

\index{to\_gxgrid() (in module pygeos.io)}

\begin{fulllineitems}
\phantomsection\label{index:pygeos.io.to_gxgrid}\pysiglinewithargsret{\code{pygeos.io.}\bfcode{to\_gxgrid}}{\emph{df}, \emph{name}, \emph{griddef}}{}
Convert Pandas DataFrame to Geosoft grid
\begin{quote}\begin{description}
\item[{Parameters}] \leavevmode\begin{itemize}
\item {} 
\textbf{\texttt{df}} -- Pandas DataFrame to be Gridded

\item {} 
\textbf{\texttt{name}} -- variable name

\item {} 
\textbf{\texttt{griddef}} -- pygslib GridDef object to set grid definition

\end{itemize}

\item[{Returns}] \leavevmode

Geosoft grid (.grd)

Note: this function does not set the projection


\item[{Return type}] \leavevmode
(GXgrd)

\end{description}\end{quote}

\end{fulllineitems}



\chapter{plot module}
\label{index:module-pygeos.plot}\label{index:plot-module}\index{pygeos.plot (module)}\index{corr\_plot() (in module pygeos.plot)}

\begin{fulllineitems}
\phantomsection\label{index:pygeos.plot.corr_plot}\pysiglinewithargsret{\code{pygeos.plot.}\bfcode{corr\_plot}}{\emph{df}, \emph{outfl=None}, \emph{color='dimgray'}}{}
Generate a matrix plot for examining pairwise relationships between datatypes
\begin{quote}\begin{description}
\item[{Parameters}] \leavevmode\begin{itemize}
\item {} 
\textbf{\texttt{df}} -- Pandas DataFrame with data

\item {} 
\textbf{\texttt{outfl}} -- file path for saved plot (optional)

\item {} 
\textbf{\texttt{color}} -- color for points on scatter plots

\end{itemize}

\end{description}\end{quote}

\end{fulllineitems}

\index{fancy\_xplot() (in module pygeos.plot)}

\begin{fulllineitems}
\phantomsection\label{index:pygeos.plot.fancy_xplot}\pysiglinewithargsret{\code{pygeos.plot.}\bfcode{fancy\_xplot}}{\emph{GridData}, \emph{WellData}, \emph{var1}, \emph{var2}, \emph{outfl}, \emph{show=True}, \emph{figsize=(15}, \emph{15)}}{}
Cross plotting function used to portray gridded data and extracted scattered data on the same plot
Useful to show relationship between grids and data actually used in training for PA methods
\begin{quote}\begin{description}
\item[{Parameters}] \leavevmode\begin{itemize}
\item {} 
\textbf{\texttt{GridData}} -- Pandas DataFrame of gridded data

\item {} 
\textbf{\texttt{WellData}} -- Scattered well data extracted from grid locations (same variable names as GridData)

\item {} 
\textbf{\texttt{var2}} (\emph{var1,}) -- variables for crossplotting

\item {} 
\textbf{\texttt{outfl}} -- filepath for saving figure

\item {} 
\textbf{\texttt{show}} -- True/False to show figure

\item {} 
\textbf{\texttt{figsize}} -- set figure size (default: 15x15)

\end{itemize}

\end{description}\end{quote}

\end{fulllineitems}

\index{log\_plot() (in module pygeos.plot)}

\begin{fulllineitems}
\phantomsection\label{index:pygeos.plot.log_plot}\pysiglinewithargsret{\code{pygeos.plot.}\bfcode{log\_plot}}{\emph{Data}, \emph{columns}, \emph{topsdict}, \emph{top}, \emph{bottom}, \emph{cmap=\textless{}matplotlib.colors.LinearSegmentedColormap object at 0x0000000006BD60F0\textgreater{}}, \emph{cutoff=None}, \emph{outfl=None}, \emph{show=True}, \emph{returnfig=False}}{}
Plot well log information from LAS file
\begin{quote}\begin{description}
\item[{Parameters}] \leavevmode\begin{itemize}
\item {} 
\textbf{\texttt{Data}} -- Pandas DataFrame with log data

\item {} 
\textbf{\texttt{columns}} -- columns within DataFrame to plot

\item {} 
\textbf{\texttt{topsdict}} -- a dictionary of tops from which top and bottom will be selected

\item {} 
\textbf{\texttt{top}} -- top formation for plot

\item {} 
\textbf{\texttt{bottom}} -- bottom formation for plot

\item {} 
\textbf{\texttt{cmap}} -- colormap (default is jet)

\item {} 
\textbf{\texttt{cutoff}} -- cutoff value (optional)

\item {} 
\textbf{\texttt{outfl}} -- filepath for saved plot (optional)

\item {} 
\textbf{\texttt{show}} -- True/False to show plot (default: True)

\item {} 
\textbf{\texttt{returnfig}} -- True/False to return figure as object, useful for color mapping / legend

\end{itemize}

\end{description}\end{quote}

\end{fulllineitems}

\index{probplt() (in module pygeos.plot)}

\begin{fulllineitems}
\phantomsection\label{index:pygeos.plot.probplt}\pysiglinewithargsret{\code{pygeos.plot.}\bfcode{probplt}}{\emph{dataSeries}, \emph{outfl=None}, \emph{color='black'}}{}
\end{fulllineitems}



\chapter{utils module}
\label{index:module-pygeos.utils}\label{index:utils-module}\index{pygeos.utils (module)}\index{gap() (in module pygeos.utils)}

\begin{fulllineitems}
\phantomsection\label{index:pygeos.utils.gap}\pysiglinewithargsret{\code{pygeos.utils.}\bfcode{gap}}{\emph{data}, \emph{refs=None}, \emph{nrefs=20}, \emph{ks=range(1}, \emph{11)}}{}
Compute the Gap statistic for an nxm dataset in data.
Either give a precomputed set of reference distributions in refs as an (n,m,k) scipy array,
or state the number k of reference distributions in nrefs for automatic generation with a
uniformed distribution within the bounding box of data.
Give the list of k-values for which you want to compute the statistic in ks.

\end{fulllineitems}

\index{getcoords() (in module pygeos.utils)}

\begin{fulllineitems}
\phantomsection\label{index:pygeos.utils.getcoords}\pysiglinewithargsret{\code{pygeos.utils.}\bfcode{getcoords}}{\emph{GridDef}}{}
Generate Pandas DataFrame of XYZ coordinates based on grid definition
\begin{quote}\begin{description}
\item[{Parameters}] \leavevmode
\textbf{\texttt{GridDef}} -- pyGSLIB Grid Definition object

\item[{Returns}] \leavevmode
Pandas DataFrame with the columns X, Y and Z

\item[{Return type}] \leavevmode
coords

\end{description}\end{quote}

\end{fulllineitems}

\index{getidx() (in module pygeos.utils)}

\begin{fulllineitems}
\phantomsection\label{index:pygeos.utils.getidx}\pysiglinewithargsret{\code{pygeos.utils.}\bfcode{getidx}}{\emph{points}, \emph{griddef}}{}
Return the index of a given point in a grid
Best used with df.apply() function in pandas
\begin{quote}\begin{description}
\item[{Parameters}] \leavevmode\begin{itemize}
\item {} 
\textbf{\texttt{points}} -- Pandas DataFrame including columns labeled X and Y

\item {} 
\textbf{\texttt{griddef}} -- pyGSLIB Grid Definition object

\end{itemize}

\item[{Returns}] \leavevmode
index value for location in grid

\item[{Return type}] \leavevmode
(int)

\end{description}\end{quote}

\end{fulllineitems}

\index{gridExtract() (in module pygeos.utils)}

\begin{fulllineitems}
\phantomsection\label{index:pygeos.utils.gridExtract}\pysiglinewithargsret{\code{pygeos.utils.}\bfcode{gridExtract}}{\emph{ptxy}, \emph{griddef}, \emph{gridDF}}{}
Uses getidx to extract gridded data based on point locations
\begin{quote}\begin{description}
\item[{Parameters}] \leavevmode\begin{itemize}
\item {} 
\textbf{\texttt{ptxy}} -- Pandas DataFrame containing X Y columns of coordinates

\item {} 
\textbf{\texttt{griddef}} -- pyGSLIB Grid Definition object

\item {} 
\textbf{\texttt{gridDF}} -- Pandas Dataframe containing grid(s)

\end{itemize}

\item[{Returns}] \leavevmode
Values extracted from grid(s) at XY locations

\item[{Return type}] \leavevmode
(Pandas DataFrame)

\end{description}\end{quote}

\end{fulllineitems}



\chapter{Indices and tables}
\label{index:indices-and-tables}\begin{itemize}
\item {} 
\DUspan{xref,std,std-ref}{genindex}

\item {} 
\DUspan{xref,std,std-ref}{modindex}

\item {} 
\DUspan{xref,std,std-ref}{search}

\end{itemize}


\renewcommand{\indexname}{Python Module Index}
\begin{theindex}
\def\bigletter#1{{\Large\sffamily#1}\nopagebreak\vspace{1mm}}
\bigletter{p}
\item {\texttt{pygeos.io}}, \pageref{index:module-pygeos.io}
\item {\texttt{pygeos.PA}}, \pageref{index:module-pygeos.PA}
\item {\texttt{pygeos.plot}}, \pageref{index:module-pygeos.plot}
\item {\texttt{pygeos.utils}}, \pageref{index:module-pygeos.utils}
\end{theindex}

\renewcommand{\indexname}{Index}
\printindex
\end{document}
